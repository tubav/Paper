\section{Problem Statement and Terminology}\label{sec:problem}
%-------------------------------------------------------------------------------
%Architecture of proposed system(s) to achieve this model should be more
%generic than your own peculiar implementation. Always include at least one
%figure.
%
%The body should contain sufficient motivation, with at least one example
%scenario, preferably two, with illustrating figures, followed by a crisp
%generic problem statement model, i.e., functionality, particularly emphasizing
%"new" functionality. The paper may or may not include formalisms. General
%evaluations of your algorithm or architecture, e.g., material proving that the
%algorithm is O(log N), go here, not in the evaluation section.
%-------------------------------------------------------------------------------


% -------------------------------------------------------------------------------
% 1. Define the problem to be considered in detail. Typically this section might
%    begin with something like: “Consider a packet radio system consisting of a
%    single central repeater surrounded by user terminals. Each user transmits
%   packets to the central repeater using a slotted ALOHA protocol[1]. The  
% transmissions from all users are assumed to be on the same frequency...” 
% The discussion should proceed in this way until the problem is completely 
% defined.
% -------------------------------------------------------------------------------
\subsection{Motivation}
Consider a system \ac{TODPM}\cite{rfc5841}.

%-------------------------------------------------------------------------------
% 2. Define all terminology and notation used. Usually the terminology and
%   notation are defined with the problem itself. 
%-------------------------------------------------------------------------------
\subsection{Terminology}
Define all terminology and notation used. Usually the terminology and notation
are defined with the problem itself.

%-------------------------------------------------------------------------------
%3. Develop the equations on which your results will be based 
%-------------------------------------------------------------------------------

Develop the equations on which your results will be based.

